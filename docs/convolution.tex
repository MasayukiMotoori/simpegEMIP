%gji_extra_guide.tex
% \documentclass{gji}
\documentclass[extra,mreferee]{gji}
\usepackage{times, color}
% \usepackage{mathrsfs,amsmath}
% \documentclass[a4paper, 11pt]{article}
% \usepackage{fullpage}
\usepackage[pdftex]{graphicx}
\usepackage{mathrsfs, amsmath, amsfonts,xspace}
\usepackage[pagewise, mathlines]{lineno}

% \usepackage{framed, color, fancybox}
\author[Seogi Kang and Douglas W. Oldenburg]
   {Seogi Kang and Douglas W. Oldenburg \\
    Department of Earth, Ocean and Atmospheric Sciences,
    University of British Columbia,
    B.C. \emph{V6T 1Z4}, Canada
  }

\title{Simulations of induced polarization effects in time domain electromagnetic data}


%% =============================================================================
%% My eqs
%% =============================================================================
\newcommand{\SimPEG}{\textsc{SimPEG}\xspace}
\renewcommand{\div}{\nabla\cdot}
\newcommand{\grad}{\vec \nabla}
\newcommand{\curl}{{\vec \nabla}\times}
\newcommand {\J}{{\vec J}}
\renewcommand{\H}{{\vec H}}
\newcommand {\E}{{\vec E}}
\newcommand{\siginf}{\sigma_\infty}
\newcommand{\dsig}{\triangle\sigma}
\newcommand{\dcurl}{{\mathbf C}}
\newcommand{\dgrad}{{\mathbf G}}
\newcommand{\Acf}{{\mathbf A_c^f}}
\newcommand{\Ace}{{\mathbf A_c^e}}
\renewcommand{\S}{{\mathbf \Sigma}}
\newcommand{\St}{{\mathbf \Sigma_\tau}}
\newcommand{\T}{{\mathbf T}}
\newcommand{\Tt}{{\mathbf T_\tau}}
\newcommand{\diag}{\mathbf{diag}}
\newcommand{\M}{{\mathbf M}}
\newcommand{\MfMui}{{\M^f_{\mu^{-1}}}}
\newcommand{\MfMuoi}{{\M^f_{\mu_0^{-1}}}}
\newcommand{\dMfMuI}{{d_m (\M^f_{\mu^{-1}})^{-1}}}
\newcommand{\dMfMuoI}{{d_m (\M^f_{\mu_0^{-1}})^{-1}}}
\newcommand{\MeSig}{{\M^e_\sigma}}
\newcommand{\MeSigInf}{{\M^e_{\sigma_\infty}}}
\newcommand{\MeSigInfEtab}{{\M^e_{\sigma_\infty \bar{\eta}}}}
\newcommand{\MeSigInfEtat}{{\M^e_{\sigma_\infty \peta}}}
\newcommand{\MedSig}{{\M^e_{\triangle\sigma}}}
\newcommand{\MeSigO}{{\M^e_{\sigma_0}}}
\newcommand{\Me}{{\M^e}}
\newcommand{\Js}{\mathbf{J}^s}
\newcommand{\Mes}[1]{{\M^e_{#1}}}
\newcommand{\Mee}{{\M^e_e}}
\newcommand{\Mej}{{\M^e_j}}
\newcommand{\BigO}[1]{\mathcal{O}\bigl(#1\bigr)}
\newcommand{\bE}{\mathbf{E}}
\newcommand{\bEp}{\mathbf{E}^p}
\newcommand{\bB}{\mathbf{B}}
\newcommand{\bBp}{\mathbf{B}^p}
\newcommand{\bEs}{\mathbf{E}^s}
\newcommand{\bBs}{\mathbf{B}^s}
\newcommand{\bH}{\mathbf{H}}
\newcommand{\B}{\vec{B}}
\newcommand{\D}{\vec{D}}
\renewcommand{\H}{\vec{H}}
\newcommand{\s}{\vec{s}}
\newcommand{\bfJ}{\bf{J}}
\newcommand{\vecm}{\vec m}
\renewcommand{\Re}{\mathsf{Re}}
\renewcommand{\Im}{\mathsf{Im}}
\renewcommand {\j}  { {\vec j} }
\newcommand {\h}  { {\vec h} }
\renewcommand {\b}  { {\vec b} }
\newcommand {\e}  { {\vec e} }
\renewcommand {\d}  { {\vec d} }
\renewcommand {\u}  { {\vec u} }

\renewcommand {\dj}  { {\mathbf{j} } }
\renewcommand {\dh}  { {\mathbf{h} } }
\newcommand {\db}  { {\mathbf{b} } }
\newcommand {\de}  { {\mathbf{e} } }

\newcommand{\vol}{\mathbf{v}}
\newcommand{\I}{\vec{I}}
\newcommand{\A}{\mathbf{A}}
\newcommand{\bI}{\mathbf{I}}
\newcommand{\bus}{\mathbf{u}^s}
\newcommand{\brhss}{\mathbf{rhs}_s}
\newcommand{\bup}{\mathbf{u}^p}
\newcommand{\brhs}{\mathbf{rhs}}
%%-------------------------------
\newcommand{\bon}{b^{on}(t)}
\newcommand{\bp}{b^{p}}
\newcommand{\dbondt}{\frac{db^{on}(t)}{dt}}
\newcommand{\dfdt}{\frac{df(t)}{dt}}
\newcommand{\dfdtdsiginf}{\frac{\partial\frac{df(t)}{dt}}{\partial\siginf}}
\newcommand{\dfdsiginf}{\frac{\partial f(t)}{\partial\siginf}}
\newcommand{\dbgdsiginf}{\frac{\partial b^{Impulse}(t)}{\partial\siginf}}
\newcommand{\digint}{\frac{2}{\pi}\int_0^{\infty}}
\newcommand{\Gbiot}{\mathbf{G}_{Biot}}
%%-------------------------------
\newcommand{\peta}{\tilde{\eta}}
\newcommand{\eFmax}{\e^{\ F}_{max}}
\newcommand{\eref}{\e^{\ ref}}
\newcommand{\jref}{\j^{\ ref}}
\newcommand{\dip}{d^{IP}}
\newcommand{\sigpert}{\delta\sigma}
\newcommand{\bzip}{b_z^{IP}}
\newcommand{\dbzdtip}{\frac{\partial b_z^{IP}}{\partial t}}


\begin{document}

\label{firstpage}

\maketitle

% Key words:
% Geomagnetic induction,
% Electromagnetic theory,
% Electrical properties,
% Numerical approximations and analyses

\begin{summary}
XXX
\end{summary}

%%=====================================================================
%% Section. Intro.
%%=====================================================================

% \linenumbers
\section{Formulation}
Complex conductivity model in Laplace domain can be expressed as
\begin{equation}
  \sigma (s) = \siginf + \dsig (s)
\end{equation}
where $s=\imath \omega$ is the Laplace transform variable. Inverse Laplace transform of $\sigma(s)$ yields:
\begin{equation}
  \mathcal{L}^{-1}[\sigma(s)] = \sigma(t) = \siginf \delta (t) + \dsig (t)
  \label{eq:sigma_time}
\end{equation}
where  $\delta(t)$ is a Dirac-Delta function and $\dsig(t) = \mathcal{L}^{-1}[\dsig(s)]$.
Maxwell's equations can be written as
\begin{equation}
  \curl \e = -\frac{\partial \b}{\partial t} \\
  \label{eq:faraday}
\end{equation}
\begin{equation}
  \curl \mu^{-1} \b - \j = \j_s \\
  \label{eq:ampere}
\end{equation}
Considering a time-dependent conductivity, Ohm's Law can be written as
\begin{equation}
  \j = \int_0^t \sigma(t-u) \e (u) du
  \label{eq:ohmslaw}
\end{equation}
And substituting Eq. \ref{eq:sigma_time} yields
\begin{equation}
  \j = \siginf \e + \int_0^t \dsig(t-u) \e (u) du
  \label{eq:ohmslaw_two}
\end{equation}
Using backward euler method, we respectively discretize Eqs. \ref{eq:faraday} and \ref{eq:ampere} in time:
\begin{equation}
  \curl \e^{\ (n)} = -\frac{\b^{(n)}-\b^{(n-1)}}{\triangle t^{(n)}}
  \label{eq:faraday_time}
\end{equation}
\begin{equation}
  \curl \mu^{-1} \b^{(n)} - \j^{(n)} = \j_s^{(n)} \\
  \label{eq:ampere_time}
\end{equation}
where $\triangle t^{(n)} = t^{(n)}- t^{(n-1)}$.
To discretize integration part in Eq. \ref{eq:ohmslaw_two}, we use trapezoidal rule:
\begin{equation}
  \int_{t^{(k-1)}}^{t^{(k)}} \dsig(t-u) \e (u) du
  = \frac{\triangle t^{(k)}}{2} \Big(\dsig (t^{(n)} - t^{(k-1)}) \e^{\ (k-1)} + \dsig (t^{(n)} - t^{(k)}) \e^{\ (k)} \Big)
\end{equation}
Hence the Ohm's Law shown in Eq. \ref{eq:ohmslaw_two} can be discretized as
\begin{equation}
  \j^{(n)} = \siginf \e^{\ (n)} +
  \sum_{k=1}^{n} \frac{\triangle t^{(k)}}{2} \Big(\dsig (t^{(n)} - t^{(k-1)}) \e^{\ (k-1)} + \dsig (t^{(n)} - t^{(k)}) \e^{\ (k)} \Big)
\end{equation}
This can be rewritten as
\begin{equation}
  \j^{(n)} = \Big(\siginf + \gamma (\triangle t^{(n)})\Big)\e^{\ (n)} + \j_{pol}^{(n-1)}
  \label{eq:ohmslaw_time}
\end{equation}
where the polarization current, $\j_{pol}^{(n-1)}$ is
\begin{align}
  \j_{pol}^{(n-1)} = \sum_{k=1}^{n-1} \frac{\triangle t^{(k)}}{2} \Big(\dsig (t^{(n)} - t^{(k-1)}) \e^{\ (k-1)} + \dsig (t^{(n)} - t^{(k)}) \e^{\ (k)} \Big) \nonumber \\
  +  \kappa(\triangle t^{(n)}) \e^{\ (n-1)}
\end{align}
Using mimetic finite volume approach, we correspondingly discretize Eqs. \ref{eq:faraday_time}, \ref{eq:ampere_time}, and \ref{eq:ohmslaw_time}
\begin{equation}
  \dcurl \de^{\ (n)} = -\frac{\db^{(n)}-\db^{(n-1)}}{\triangle t^{(n)}}
    \label{eq:faraday_discrete}
\end{equation}
\begin{equation}
  \dcurl \MfMui \db^{(n)} - \Me \dj^{(n)} = \mathbf{s}_e^{(n)} \\
  \label{eq:ampere_discrete}
\end{equation}
\begin{equation}
  \Me\dj^{(n)} = \Mes{A}^{(n)}\de^{\ (n)} + \dj_{pol}^{(n-1)}
  \label{eq:ohmslaw_discrete}
\end{equation}
where
\begin{align}
  \dj_{pol}^{(n-1)} = \sum_{k=1}^{n-1} \frac{\triangle t^{(k)}}{2} \Big(\Mes{\dsig (n, k-1)} \e^{\ (k-1)} + \Mes{\dsig (n, k)} \de^{\ (k)} \Big) \nonumber \\
  +  \Mes{\kappa} \de^{\ (n-1)}
\end{align}
We can rearrange above equations to solve either $\de$ or $\db$. First consider solving $\de$ hence we remove $\db$.
\begin{align}
  \Big(\dcurl^T \MfMui \dcurl + \frac{1}{\triangle t^{(n)}} \Mes{A}^{(n)}\Big) \de^{(n)} \nonumber \\
  = - \frac{1}{\triangle t^{(n)}} (\mathbf{s}_e^{(n)}-\mathbf{s}_e^{(n-1)})
    + \frac{1}{\triangle t^{(n)}} \Me \dj^{(n-1)} - \frac{1}{\triangle t^{(n)}} \dj_{pol}^{(n-1)}
\end{align}



\end{document}

